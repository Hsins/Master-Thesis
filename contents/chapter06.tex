% !TeX root = ../main.tex

\chapter{結論與建議}

\section{研究總結}

本研究以虛擬電廠 (Virtual Power Plant, VPP) 概念整合風力電場與電動汽車,透過預測風速配合風機功率模型取得風力機組發電數量後,考慮電動汽車作為儲能設備來參與電力市場交易,並透過模型預測控制 (Model Predictive Control, MPC) 在有限時域下針對動態變化的虛擬電廠進行儲能調度來獲得最佳收益。主要工作包括:

% \begin{enumerate}[label = (\arabic*)]
\begin{itemize}
  \item 提出使用小波轉換分解風速時間序列的 SVR-ARIMA 組合預測模型,用以提高收益模型中短期風速的預測能力,使其能更準確地估算風力發電數量。
  \item 以虛擬電廠概念整合風力電場與電動汽車作為儲能設備,建構其參與電力市場的收益模型。
  \item 採用模型預測控制方法,在有限時域下針對動態變化的虛擬電廠進行調度,根據結果進行收益分析與討論。
\end{itemize}

案例分析中,本研究採用我國\uline{澎湖}地區\uline{東吉島} (DONGJIDAO) 測站的歷史風速作為風速預測模型之驗證,並作為後續收益分析之依據。案例分析結果提供我們以下結論:

\begin{itemize}
  \item 短期風速預測中,採用小波分解 ARIMA-SVR 組合預測模型,透過平均絕對誤差 (Mean Absolute Error, MAE) 和均方根誤差 (Root Mean Squares Error, RMSE) 進行預測能力評價,顯示相較於單一 ARIMA 預測模型或單一 SVR 預測模型能提高預測能力。
  \item 整合電動汽車作為儲能設備,相較於沒有使用電動汽車作為儲能設備的虛擬電廠有著更高的收益;在本研究所模擬的情境下,最高能有 $15.41\%$ 的收益成長。
  \item 本文所建構之模型可用於估算虛擬電廠最佳收益,以及為達最佳收益所需的電動汽車數量。在本研究所模擬的情境下,假設電動車電池放電深度為 $\text{DoD}=0.25$ 時,電動汽車數量需求範圍約在 $1,721$ 輛至 $5,651$ 輛之間。
\end{itemize}

\section{研究限制}

受限於實務上的困難,本研究仍有一些不足之處。以下補充說明本研究存在的限制條件以及未來可以改善的方向:

\begin{enumerate}[label = (\arabic*)]
  \item 案例分析中作為短期風速預測依據的觀測風速採用氣象測站資料,與實際風場內的風速之間存在差異,無法真實反映場址的預測發電數量;若能取得並參考場址內的歷史風速資料,應能更準確地預測風力發電數量。
  \item 受限於氣象資料的觀測時間,本研究中採用一小時作為調度間隔,無法兼顧因再生能源的發電數量在短時間內有劇烈變化的情境;未來可以根據資料狀況縮小調度的時間間隔,然而對於計算設備規格要求較高,且電動汽車行為較不容易於短時間內進行控制。
  \item 研究中假設虛擬電廠中的電動汽車在時段內的行為是可以被控制的,但真實世界中每位車主的行為有太多變數存在;盼未來能有更佳的方法對這部分進行量化分析。
  \item 研究中假設虛擬電廠在販售電力資源至電力市場時不需要考慮市場需求,但真實世界中可能會遭遇特定季節或時段出現電力過剩的狀況。
  \item 科技日新月異加上政策並非一成不變,比如案例分析中所使用的電動汽車與時間電價數據,可能在未來有大幅度的變動,本文結論亦需進行滾動式修正。
\end{enumerate}

欲以有限的模型描述現實中萬千的變化幾乎無法做到,因此本研究上仍有許多限制與改善之處無法一一陳列,僅期望透過有限的資料在所能描述的範疇內儘可能地表達看法。

% \section{結語}

% 能源議題不僅與