% !TeX root = ../main.tex

\chapter{結論與建議}

本章為結論與建議,結論部分將對本研究所獲得之結果進行歸納與總結,建議部分將綜合研究限制與未來研究進行說明。

\section{結論}

本研究以虛擬電廠(Virtual Power Plant, VPP)概念整合風力電場與電動汽車,以預測風速配合風機功率模型取得風力機組發電數量後,考慮電動汽車作為儲能設備並依據時間電價進行調度規劃來參與電力市場交易,並透過模型預測控制(Model Predictive Control, MPC)在有限時域下針對動態變化的虛擬電廠進行儲能調度來獲得最佳收益。主要工作包括:(1) 提出基於小波轉換分解風速時間序列訊號的 SVR-ARIMA 組合預測模型,據以預測短期風速 (2) 以虛擬電廠概念整合風力電場與電動汽車,建構其參與電力市場的日前調度模型 (3) 採用模型預測控制方法在有限時域下針對動態變化的虛擬電廠進行調度,評估納入電動汽車前後的收益狀況。

依據案例分析中採用\uline{東吉島}(DONGJIDAO)測站歷史風速進行風速預測與收益分析之結果,有以下結論:

\begin{enumerate}
  \item 使用小波轉換對風速時間序列進行訊號分解時,選用 db4 小波母函數能夠在訊號重構後得到較小的誤差。
  \item 採用 ARIMA 預測模型與 SVR 預測模型進行短期風速預測皆能獲得不錯的效果,範例計算顯示預測結果較真實結果約偏離 $5.0$\% 至 $6.0$\% 之間。
  \item 透過小波分解時間序列訊號,對細節分量與近似分量分別採用 ARIMA 預測模型與 SVR 預測模型後,進行訊號重構的組合預測相較於單一預測模型能夠小幅度提升預測準確度。
\end{enumerate}

% 雖本研究的假設多少影響評估結果,且評估結果可
% 能受到質疑,但本研究確實建立一套考量風速、系統隨機失效及各種不確定性因
% 素之離岸風力發電場投資風險評估模型,可作為政府或相關廠商之決策參考工
% 具。

\section{建議}


