% !TeX root = ../main.tex

% 中文摘要
% Chinese Abstract
\begin{abstract}

  由於近年來持續成長的電力需求以及民眾環保意識的抬頭,加上傳統集中式電力系統的缺點日漸浮現,各國紛紛開始關注能充分利用再生能源且具有良好環境效益的分散式發電 (Distributed Generation, DG) 技術,也致使虛擬電廠 (Virtual Power Plant, VPP) 概念的生成。本研究以一整合風力電場並以電動汽車做為儲能裝置的虛擬電廠為分析探討對象,以時間電價 (Time-of-Use, TOU) 做為調度規劃基礎,參與電力市場交易。為考量實際系統變化調整控制參數,本研究選擇使用模型預測控制 (Model Predictive Control, MPC),針對發電數量具動態變化的風力電場,規劃其移動時域 (Receding Horizon) 內的儲能裝置容量以獲得最佳收益;而為鼓勵將電動車併入虛擬電廠參與電力市場交易,本研究提出以電動車額外充電作為紅利,取代對電動車車主直接的金錢補助;針對短期風速的預測,本研究透過小波轉換分解時間序列訊號,並結合傳統時間序列預測與支持向量迴歸預測的一種 ARIMA-SVR 組合預測模型,讓我們得以根據現有風速資料預測風力電場的發電量。在案例模擬中,本研究使用\uline{澎湖}縣\uline{東吉島}氣象站 2018  年間的歷史風速資料,經短期風力預測與收益模型分析後,驗證所提最佳化收益分析方法的可行性。分析結果顯示,針對短期風速的預測,採用小波分解 ARIMA-SVR 組合預測模型相較於傳統單一預測模型,在預測能力上有所提升;而整合電動汽車作為儲能裝置的虛擬電廠確能獲得較高收益。

\end{abstract}

% 英文摘要
% English Abstract
\begin{abstract*}

  In recent years, we have witnessed a growing demand for electricity, higher public awareness of environmental protection, and the drawbacks of traditional centralized power systems. As a result, many countries have turned their eyes to the distributed generation (DG) technology, which makes full use of renewable energy while bringing environmental benefits. The concept of virtual power plants (VPPs) has also emerged. This research studies a VPP that contains a wind farm and uses electric vehicles (EVs) as energy storage devices. The purpose is to examine its potential to participate in the power market transaction. The time-of-use (TOU) tariff is considered the basis for power dispatching and planning. This research uses model predictive control (MPC) to real-time adjust the control parameters when a constructed analytical system evolves. It can plan the capacity of energy storage devices of the wind farm with varying generating capacity within the receding horizon and help achieve the highest revenue. To promote the integration of EVs into the VPP and the participation of VPP in power market transactions, this research proposes the replacement of cash subsidies to EV owners with additional charging benefits for their EVs. In terms of short-term wind speed forecasting, this research also adopts the method of wavelet transform to decompose time-series signals and combines the traditional time series forecasting model with a support vector regression forecasting model called ARIMA-SVR, to predict the generating capacity based on the up-to-date information about the wind speed. In the case simulation, this research predicts the short-term wind speed, analyzes the revenue model, and validates the feasibility of the proposed strategy by using the historical wind speed data of Dongjidao Weather Station of Penghu in 2018. The analytical result of this research shows that, in terms of short-term wind speed forecasting, the combined model consisting of wavelet transform and ARIMA-SVR achieves more robust forecasting capability than the traditional singular model. In addition, the use of EVs as energy storage devices can indeed bring higher revenue for the VPP.

%   In view of the ever-increasing power requirement in the recent year and the public's environmental awareness raises its head, with the addition of the defects of the traditional centralized power system is emerging progressively, each country successively starts to pay attention to Distributed Generation (DG), the technology can make use of the renewable energy and owns the friendly environmental benefits, it also gives rise to the form of the Virtual Power Plant (VPP) concept. In this study, a virtual power plant integrating wind field and using electric vehicles as energy storage devices is analyzed, to take electricity price based on Time-of-Use (TOU) rate plan as the basis of scheduled planning, to take the part in power market exchange. for the purpose of adjusting the control parameters in real time, in accordance with actual system changes, this research chooses to use Model Predictive Control (MPC) to plan the capacity of energy storage devices in the receiving horizon of wind field with dynamic generation quantity to obtain the best benefits; To encourage the integration of electric vehicles into the virtual power plant to participate in the electricity market transactions, this study proposes to use the extra charging of electric vehicles as a bonus instead of the direct monetary subsidy to the owners of electric vehicles; For short-term wind speed prediction, this research decomposes the time-series signal through wavelet transform, and combines traditional time series prediction and support vector regression prediction of ARIMA-SVR combination prediction model, which enables us to predict the number of wind power generation according to the existing wind speed data. In the case simulation, this research uses the historical wind speed data of Dongjidao Weather Station in Penghu in 2018 and verifies the feasibility of the proposed optimal revenue analysis method after short-term wind prediction and revenue model analysis. The analysis results show that, compared with the traditional single prediction model, the wavelet decomposition ARIMA-SVR combined prediction model can improve the prediction ability of short-term wind speed; The virtual power plant integrated with electric vehicles as energy storage device can get higher income.

\end{abstract*}
