% !TeX root = ../main.tex

\begin{acknowledgement}

% -- 前言 --
曾經想過許多次動筆撰寫謝辭時的情境,原以為會是在結束繁忙工作的週末下午,帶著完成學位論文的雀躍心情振筆疾書,沒想到卻是在平日夜深人靜的清晨時分,懷著不善道別的複雜愁緒搔首踟躕,而在離愁之外的則是滿滿的感謝。

% -- 指導教授 --
感謝我的指導教授 ------ 吳文方老師。感謝老師在學術研究上的指導,每當遇到研究上的困難與疑惑時,老師總是願意撥冗傾聽並給予適切的建議;感謝老師在為人處事上的教誨,不論在課堂、報告還是聚餐等場合,老師都經常分享自己過去的經驗以供借鑑;不論是研究內容、論文排版、用字遣詞、工作甚至生活上,老師都提供了相當多的幫助,不時也會天外飛來一筆地道出詼諧幽默的話語,總是能讓實驗室的大家會心一笑,舒緩同學們求學與工作上的焦慮。

% -- 口試委員 --
感謝各位口試委員 ------ 黃奎隆老師與陳湘鳳老師。感謝老師們在百忙之中仍抽空擔任口試委員,細心審閱的同時,亦提供諸多寶貴的意見以及指正,使學生的論文能夠更臻完善。

% -- 朋友 --
感謝實驗室的夥伴 ------ 運政、傅堯、博文、柏璠、庭豪、瑞亨、鈺衡、承鴻、建君、沅佑、名原、靖雯、逸家、翰飛、喬凱、文祥、雋承、宇正、鎮澤、力中、庭臻、舒婷、有朋與綾蓁。可靠度工程實驗室就像一個大家庭,感謝大家在每周報告時的討論建議,以及一起修課時的互相幫助;除了學業之外,冬至的桌遊活動、期末聚餐夜唱,還有每次報告結束後短暫的閒話家常,都讓實驗室增添許多歡樂,這些都是碩士生涯中的美好回憶。

% -- 家人 --
感謝我親愛的家人 ------ 父親彭振達先生、母親廖麗華女士以及新喻和詩涵兩位姊姊。感謝父母從小提供良好的環境讓我們能無憂地成長,無論是在學業上的栽培亦或是興趣上的發展;感謝姐姐們的保護與照顧,成長過程中慶幸能有妳們的關懷與鼓勵,這些都是煎熬困頓之際,支持我堅持下去的動力來源;感謝您們體諒身在異鄉的我,由於求學與工作的關係不能時常返家探望;感謝您們包容我那些因為年少無知而曾經做出的種種任性決定。我始終知曉這些歲月靜好的背後,都是您們在替我負重前行……

% -- 結尾 --
北周的著名文學家庾信在其作品《徵調曲》中寫道「落其實者思其樹,飲其流者懷其源」,但這一路上受到的幫助實在太多,篇幅有限無法讓我朔及各個曾經駐足的源泉,此處即使用再多的言語也無法深刻表達內心的感謝。最後將要劃上謝辭的句號,似乎也意味著這一階段的旅程即將告一段落,但人生中仍有許多故事等待完成,現在該是出發的時候了!

\begin{flushright}
新翔 謹誌於 \\[0pt]
國立臺灣大學 \\[0pt]
中華民國~一百一十~年~十~月
\end{flushright}

\end{acknowledgement}